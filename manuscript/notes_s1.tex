\documentclass{myreport}

% References
\bibliographystyle{copernicus.bst}

\begin{document}
\pagestyle{headings}

% Change figure (table, section) numbering (e.g., from 'Figure 1' to 'Figure S1')
\renewcommand{\thefigure}{\arabic{figure}}
\renewcommand{\thetable}{\arabic{table}}
\renewcommand{\thesection}{Notes S\arabic{section}}
\renewcommand{\theequation}{\arabic{equation}}

% Document must include
% ---------------------
% 

%% Title
\title{SUPPLEMENTARY INFORMATION:\\
Empirical evidence and theoretical understanding of ecosystem carbon and nitrogen cycle interactions}
\author{Benjamin D. Stocker, Ning Dong, Evan A. Perkowski, Pascal D. Schneider,
Huiying Xu, Hugo de Boer, Karin T. Rebel, Nicholas G. Smith, Kevin Van Sundert, Han Wang, Sarah E. Jones, I. Colin Prentice and Sandy P. Harrison} 

% \maketitle

\section{Analysis of modelled land C sink trends}

We evaluated the time series of the simulated and observations-based land C balance, its decadal mean for years 2011-2020 and long-term trend for years 1959-2020 from outputs of the Trends and Drivers of Terrestrial Sources and Sinks of Carbon Dioxide (TRENDY) version 8 model intercomparison \citep{sitch_trends_2024}. We downloaded the original file \texttt{Global\_Carbon\_Budget\_2021v1.0.xlsx} (\url{doi:10.18160/gcp-2021}) from the Global Carbon Budget 2021 website and exported the tabs `Terrestrial Sink' and `Global Carbon Budget' for further analysis. From the latter, we derived the land C sink as the budget residual as quantified by \citep{friedlingstein22essd}:
\begin{equation}
S_\text{Land} = (E_\text{FF} + E_\text{LUC}) - (G_\text{atm} + S_\text{ocean} + S_\text{cement}) \;,
\end{equation}
where $S_\text{Land}$ is the land sink (`Observations' in Fig. 1 of the main text), $E_\text{FF}$ are emission from fossil fuel combustion, $E_\text{LUC}$ are emissions from land use change, $G_\text{atm}$ is the atmospheric growth rate, $S_\text{ocean}$ is the ocean sink, and $S_\text{cement}$ is the C sink from cement carbonation. The land sink simulated by models was taken as the annual C flux numbers provided in the tab `Terrestrial Sink' of the original file. It was taken by \citep{friedlingstein22essd} as the global biome producitivity (net terrestrial C balance) from simulations (TRENDY S2) forced by observed CO$_2$ and climate and with constant pre-industrial land use. The identification of models into C-only and C-N coupled models was done based on information provided in Table A.1 in \citep{friedlingstein22essd}. 

%%%%%%%%%%%%%%%%%%%%%%%%%%%%%%%%%%%%%%%%%%%%%%%%%%%%%%%%%%%%%%%%%%%%%%%%%%%
\clearpage
\bibliography{references_cnreview.bib}


\end{document}

